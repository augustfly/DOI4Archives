\documentclass[11pt,a4paper]{ivoa}
\input tthdefs
\input gitmeta

\usepackage{todonotes}


\title{Best practices for the creation of and metadata for digital object identifiers in astronomy archives}

% see ivoatexDoc for what group names to use here; use \ivoagroup[IG] for
% interest groups.
\ivoagroup[IG]{Data Curation and Preservation}

\author[https://orcid.org/0000-0003-0666-6367]{August Muench}
\author[]{Gilles Landais}
\author[]{Raffaele D'ABrusco}

\editor{TBD}

% \previousversion[????URL????]{????Concise Document Label????}
\previousversion{This is a draft}


\begin{document}
\todo{This is only an outline! Everything is still to be done.}

\begin{abstract}
Many astronomy archives are producing digital object identifiers (DOI) for datasets and services.
This document aims to summarize current workflows for creating and using DOIs, 
diagnose issues in the metadata of extant DOIs, 
and develop best practices for worklows and metadata for future DOI deployment.
This note is focused on archives in Astronomy, Planetary Science, and Heliophysics. 
Additional domains may be considered at a later time.
\end{abstract}


\section*{Acknowledgments}

???? Or remove the section header ????

\section*{Conformance-related definitions}

The words ``MUST'', ``SHALL'', ``SHOULD'', ``MAY'', ``RECOMMENDED'', and
``OPTIONAL'' (in upper or lower case) used in this document are to be
interpreted as described in IETF standard RFC2119 \citep{std:RFC2119}.

The \emph{Virtual Observatory (VO)} is a
general term for a collection of federated resources that can be used
to conduct astronomical research, education, and outreach.
The \href{https://www.ivoa.net}{International
Virtual Observatory Alliance (IVOA)} is a global
collaboration of separately funded projects to develop standards and
infrastructure that enable VO applications.


\section{Introduction}

\begin{enumerate}
\item General observations. 
We focus on roles of DOIs rather than the full landscape of FAIR data \citep[e.g.,][]{Wilkinson2016}. 
It is beyond the scope of this document to address all aspects of FAIRness.
We think that this document may help guide archives wrt F.X, A.Y of 
\item Describe use cases for DOIs in astronomy: citation, provenance, ...
\item DOIs do not lead to data and other future problems
\item outline of IVOA note
\end{enumerate}

\begin{admonition}{Note}
DOIs do not lead to data, or what they do lead to is widely varied
\end{admonition}

\subsection{Role within the VO Architecture}

\begin{figure}
\centering

% As of ivoatex 1.2, the architecture diagram is generated by ivoatex in
% SVG; copy ivoatex/archdiag-full.xml to role_diagram.xml and throw out
% all lines not relevant to your standard.
% Notes don't generally need this.  If you don't copy role_diagram.xml,
% you must remove role_diagram.pdf from SOURCES in the Makefile.

%\includegraphics[width=0.9\textwidth]{role_diagram.pdf}
\caption{Architecture diagram for this document}
\label{fig:archdiag}
\end{figure}

Fig.~\ref{fig:archdiag} shows the role this document plays within the
IVOA architecture \citep{2021ivoa.spec.1101D}.


\section{Extant use of Digital Object Identifiers in Archives}

\subsection{Collections}
Examples of collection-type DOIs include MAST, Chandra, and VAMDC instances. 
Collection-type DOIs. 
There are strong variations in the metadata of Collection type DOIs.
There is also variation in the expected use cases for these DOIs. 

For example, MAST and Chandra DOIs collect dataset identifiers in their respective databases.
There is no expectation that this collection DOIs would ever themselves collection attribution (citations).
However, VAMDC collection DOIs, are intended to distributed credit to the collection of 
They attempt to distribute this citation/credit to the collection of resources by "citing" all the related resources in the DOI record.
We note that while this is the apparent intent of their records on Zenodo.
However it is not having the functional outcome bc of how the metadata is being deposited at Datacite by Zenodo. 
See Appendix A (on Datacite deposit curiosities).

\subsection{Datasets}
Examples of dataset DOIs include the IPAC, ESA, PDS, etc. 
This is the most general use of DOIs in repositories.
It is consistent with the usage of DOIs by institutional repositories, generalist repositories (e.g., Zenodo), etc.
The dataset DOI resolves to a landing page describing the dataset.
It provides one or more links to the data.
The content of those data links is not prescribed. 

\subsection{Services}
Examples of DOIs for services include IRSA (DUST).
Service DOIs lead to query tools.
They may lead to query results -- I would need examples.

Sometimes Collection DOIs act like Service DOIs but they are not.
Collection DOIs may result from queries performed at a Service.
\textbf{Warning: weeds.}

\subsection{Knowledgebases}

Define: a knowledgebase is a collection of material from many discrete sources.
All of the values contained in a knowledge base have a provenance traced to other resources
and have been curated into a single resource for reuse.
Examples include: Simbad, NEA, NED. 
DOIs for Knowledgebases lead users to the landing page of the collated, curated resource. 
DOIs do not lead to individual values etc. 
Services that return DOIs for queries against knowledgebase are considered Service DOIs (See Section \#\#).
\textbf{Warning: weeds.}

\subsection{IVOA Registry}
DOIs are being used in some manner by the Registry.
This includes both the storage of DOis and
the minting of DOIs for Registry entries.
\textit{Until I understand this better, I consider this category distinct from any of the others. 
One reason is that I fear that there is a potential mis-management of DOI services here.
The publisher of a DOI is responsible for the upkeep of the digital resource.
If the IVOA DOI minting is providing DOIs for resources it does not manage, then this is a mismanagement of DOI services.}

\section{Use cases for DOI metadata and current pathologies}

\subsection{Use Cases}
\begin{enumerate}
\item Citation \& Attribution
\item Tracking reuse and use cases
\item Provenance (?)
\item IVOA specific desires if different from above.
\item Reproducibility (but not really)
\end{enumerate}


\subsection{Pathologies}
The primary pathology evident in DOI minting today is a mismatch in the metadata created and the use case desired. 
Succinctly, the metadata supplied by astronomy archives is often insufficient to ensure the accurate citation of the datasets used. 

\begin{enumerate}
\item Incomplete metadata: missing authors, terrible titles, misunderstood dates 
\item Inconsistent: transmutations of metadata between systems lead to inconsistent metadata
\item Versioning is ill defined and often opaquely transmitted
\item DOIs do not lead to data, or what they do lead to is widely varied
\end{enumerate}


\section{Best Practices for DOI Workflows}

\subsection{Workflows: Collections}
\subsection{Workflows: Datasets}
\subsection{Workflows: Knowledgebases}
\subsection{Workflows: Other}

\section{Best Practices for DOI Metadata}

\subsection{Core Metadata: Names, Dates, }
\begin{enumerate}
\item Authorship: Decide on what mechanism will be used to transmit authorship (individuals or collaboration/institute) and stick to it.
\item Authorship: The authorship list must include all contributors to the creation of a resource, individually, ... or none of them.
\item Title: the title of a deposit must be unique and should describe the contents of the resource.
\item Version: heaven help us.
\item Dates: The date describes when the resource was made available on line, not when the DOI was minted. 
The date of the most recent update to the DOI contents and/or metadata is a different date.
\item Description: Consider if ADS indexes your resource. If you do not provide a description then the resource will be indexed w/o any text/abstract!
\item Licensing and reuse permissions:
\item Transmutations of metadata: 
If you are supplying metadata in multiple formats (on landing pages, in DataCite XML, in schema.org tags)
then ensure that all forms are consistently transformed. 
There are crosswalks between these metadata schema (so list them)
\end{enumerate}

\subsection{Relationship Metadata}
\begin{enumerate}
\item 
\end{enumerate}


\appendix
\section{Catalog of Repositories issuing DOIs}

\section{Important DataCite (or other) }
\section{Changes from Previous Versions}

No previous versions yet.
% these would be subsections "Changes from v. WD-..."
% Use itemize environments.


% NOTE: IVOA recommendations must be cited from docrepo rather than ivoabib
% (REC entries there are for legacy documents only)
\bibliography{ivoatex/ivoabib,ivoatex/docrepo,DOI4Archives}


\end{document}
