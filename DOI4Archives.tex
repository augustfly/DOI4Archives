\documentclass[11pt,a4paper]{ivoa}
\input tthdefs
\input gitmeta

\usepackage{todonotes}


\title{Best practices for the creation of and metadata for digital object identifiers in astronomy archives}

% see ivoatexDoc for what group names to use here; use \ivoagroup[IG] for
% interest groups.
\ivoagroup[IG]{Data Curation and Preservation}

\author[https://orcid.org/0000-0003-0666-6367]{August Muench}
\author[]{Gilles Landais}
\author[]{Raffaele D'ABrusco}
\author[]{Anne Raugh}

\editor{TBD}

% \previousversion[????URL????]{????Concise Document Label????}
\previousversion{This is a draft}


\begin{document}
\todo{This is only an outline! Everything is still to be done.}

\begin{abstract}
Many astronomy archives are producing digital object identifiers (DOI) for datasets and services.
This document aims to summarize current workflows for creating and using DOIs, 
diagnose issues in the metadata of extant DOIs, 
and develop best practices for worklows and metadata for future DOI deployment.
This note is focused on archives in Astronomy, Planetary Science, and Heliophysics. 
Additional domains may be considered at a later time.
\end{abstract}


\section*{Acknowledgments}

???? Or remove the section header ????

\section*{Conformance-related definitions}

The words ``MUST'', ``SHALL'', ``SHOULD'', ``MAY'', ``RECOMMENDED'', and
``OPTIONAL'' (in upper or lower case) used in this document are to be
interpreted as described in IETF standard RFC2119 \citep{std:RFC2119}.

The \emph{Virtual Observatory (VO)} is a
general term for a collection of federated resources that can be used
to conduct astronomical research, education, and outreach.
The \href{https://www.ivoa.net}{International
Virtual Observatory Alliance (IVOA)} is a global
collaboration of separately funded projects to develop standards and
infrastructure that enable VO applications.


\section{Introduction}

\begin{enumerate}
\item General observations. 
We focus on roles of DOIs rather than the full landscape of FAIR data \citep[e.g.,][]{Wilkinson2016}. 
It is beyond the scope of this document to address all aspects of FAIRness.
We think that this document may help guide archives with achieve compliance for F.X, A.Y of the specific .
\item Describe use cases and problems for DOIs in astronomy: citation, provenance, ...
\item DOIs do not lead to data and other future problems
\item Outline of IVOA note
\item A complete listing of all astronomy/heliophysics/planetary science archvies issuing dataset (or other forms of) DOIs is given in Appendix A
\item A series of important example case studies in DataCite metadata deposits is given in Appendix B.
\end{enumerate}

\begin{admonition}{Note}
Dataset DOIs do not lead to data, or what they do lead to is widely varied.
\end{admonition}


\subsection{Role within the VO Architecture}

\begin{figure}
\centering

% As of ivoatex 1.2, the architecture diagram is generated by ivoatex in
% SVG; copy ivoatex/archdiag-full.xml to role_diagram.xml and throw out
% all lines not relevant to your standard.
% Notes don't generally need this.  If you don't copy role_diagram.xml,
% you must remove role_diagram.pdf from SOURCES in the Makefile.

%\includegraphics[width=0.9\textwidth]{role_diagram.pdf}
\caption{Architecture diagram for this document}
\label{fig:archdiag}
\end{figure}

Fig.~\ref{fig:archdiag} shows the role this document plays within the
IVOA architecture \citep{2021ivoa.spec.1101D}.


\section{Extant use of Digital Object Identifiers in Archives}

\subsection{Collections}
\label{sec:intro:collections}


Collection-type DOIs direct users to a collection of individual data records in a particular archive.
Examples of collection-type DOIs include the DOI services provided by the Mikulski Archive for Space Telescopes (MAST)\footnote{\url{https://archive.stsci.edu}} \citep{2018ApJS..236...20N}, Chandra Data Archive (CDA)\footnote{\url{https://cxc.harvard.edu/cda/}} \citep{2018EPJWC.18612011R}, and the VAMDC Consortium\footnote{\url{https://vamdc.org}} Query Store service \citep{2018Galax...6..105M}.  
There is significant variation in the expected use cases for these Collection DOIs. 
There are also strong variations in the metadata of Collection-type DOIs.

Considering use-cases, both MAST and Chandra Collection DOIs collect dataset identifiers within their respective databases.
In both cases there is no expectation that these Collection DOIs would ever themselves collect attribution (i.e., be cited in the reference list of a corresponding Journal article).
However, VAMDC Collection DOIs, are intended to collect and distribute credit to the collection of database identifiers they contain.
They attempt to distribute this citation/credit to the collection of resources by "citing" all the related resources in the saved VAMDC query record.
Because this depends upon the capabilities of both the chosen DOI minting service (Zenodo) and the DataCite Schema, additional  notes on the details and the outcomes of this effort is provided in one of our case studies in Appendix B. 

Similarly the metadata of Collection DOIs vary between MAST and CDA examples. 
As described in \citet{2023ChNew..34....5D}, Chandra Data Collection DOIs create complete records of individual using \texttt{relatedIdentifier} tags and predicates. 
The metadata of MAST Collection DOIs do not provide detailed information on the individual MAST records collated in the collection.


\subsection{Datasets}
\label{sec:intro:datasets}

Examples of dataset DOIs include the IPAC, ESA, PDS, MAST etc. 
The most general use of DOIs in repositories.
It is consistent with the usage of DOIs by institutional repositories, generalist repositories (e.g., Zenodo), etc.
The dataset DOI resolves to a landing page describing the dataset.
The landing page provides one or more links to data.
The content and structure of those data links is not prescribed. 

Use cases 

\subsection{Services}
\label{sec:intro:services}

Examples of DOIs for services include IRSA (DUST).
Service DOIs lead to query tools.
They may lead to query results -- I would need examples.

Sometimes Collection DOIs act like Service DOIs but they are not.
Collection DOIs may result from queries performed at a Service.
\textbf{Warning: weeds.}

\subsection{Knowledgebases}
\label{sec:intro:kdbs}

A knowledgebase is a collection of material collated from many discrete sources.
All of the values contained in a knowledgebase have a provenance traced to other resources and have been curated into a single database for reuse.
Examples include: Simbad \citep[as originally described in, ][]{2000A&AS..143....9W}, NASA Exoplanet Archive (NEA) \citep{NEA12-doi2bib} \citep[as originally described in,][]{2013PASP..125..989A}, NASA Extragalactic Database (NED) \citep{NED1-doi2bib} \citep[as originally described in,][]{1991ASSL..171...89H}. 

Current observations about DOIs for knowledgebases include:
DOIs for knowledgebases lead users to interstitial landing pages rather than directly to the collated, curated resources.
DOIs for knowledgebases do not lead to individual values, e.g., the results of a query against that knowledgebase.
DOIs for knowledgebases never provide information in their metadata about the \textit{state} of a knowledgebase: its current version; last update; etc. 
Nor is this information on the interstitial landing pages of knowledgebases.
Services that return DOIs for queries against knowledgebase are considered Service DOIs (See Section~\ref{sec:intro:services}).
\textbf{Warning: weeds.}

\subsection{IVOA Registry}
DOIs are being used in some manner by the Registry.
This includes both the indexing of DOIs related to Registry records and the minting of DOIs for Registry entries.
\textit{Until I understand this better, I consider this category distinct from any of the others.  One reason is that I fear that there is a potential mis-management of DOI services here. The publisher of a DOI is responsible for the upkeep of the digital resource. If IVOA DOI minting is providing DOIs for resources it does not manage, then this is potential mismanagement of DOI services. It is not unlike the ill-begotten aims of similar 3rd party DOI minting by \url{Re3data.Org} and \url{FAIRsharing.org} that have, for example, created DOIs for Simbad \citep{https://doi.org/10.17616/r39w29,https://doi.org/10.25504/fairsharing.rd6gxr} that really ought never be used by anyone aiming to cite Simbad for the source of their results. }

\section{Use cases for DOI metadata and current pathologies}

\subsection{Use Cases}
\begin{enumerate}
\item Citation \& Attribution
\item Tracking reuse and discovering reuse modes
\item Provenance (?)
\item IVOA specific desires if different from above.
\item Reproducibility (but not really)
\end{enumerate}


\subsection{Pathologies}
The primary pathology evident in DOI creation today is a mismatch between the metadata created and the use case desired. 
Succinctly, the metadata supplied by astronomy archives is often insufficient to ensure the accurate citation of the datasets created.
By detailing these pathologies and triaging their less-than-desirable outcomes we can aim to develop empirically-defined best practices to guide repositories forward with the use of identifiers.
Here is a topical list of pathologies:

\begin{enumerate}
\item Incomplete metadata: missing authors, generic or misstated titles, misunderstood dates;
\item Inconsistent metadata: transmutations of metadata between systems lead to inconsistent metadata deposits;
\item (Un)versioned data: versioning is mostly nonexistent and when provided it is ill defined and often opaquely transmitted;
\item Misconceptions: DOIs do not lead to data, or what they do lead to is widely varied.
\end{enumerate}


\section{Best Practices for DOI Workflows}

\subsection{Workflows: Collections}
\subsection{Workflows: Datasets}
\subsection{Workflows: Knowledgebases}
\subsection{Workflows: Other}

\section{Best Practices for DOI Metadata}

\subsection{Core Metadata: Names, Dates, }
\begin{enumerate}
\item Authorship: Decide on what mechanism will be used to transmit authorship (individuals or collaboration/institute) and stick to it.
\item Authorship: The authorship list must include all contributors to the creation of a resource, individually, ... or none of them.
\item Title: the title of a deposit must be unique and should describe the contents of the resource.
\item Version: heaven help us.
\item Dates: The date describes when the resource was made available on line, not when the DOI was minted. 
The date of the most recent update to the DOI contents and/or metadata is a different date.
\item Description: Consider if ADS indexes your resource. If you do not provide a description then the resource will be indexed w/o any text/abstract!
\item Licensing and reuse permissions:
\item Transmutations of metadata: 
If you are supplying metadata in multiple formats (on landing pages, in DataCite XML, in schema.org tags)
then ensure that all forms are consistently transformed. 
There are crosswalks between these metadata schema (so list them)
\end{enumerate}

\subsection{Relationship Metadata}
\begin{enumerate}
\item 
\end{enumerate}


\appendix
\section{Catalog of Repositories issuing DOIs}

\section{Important DataCite (or other) }
\section{Changes from Previous Versions}

No previous versions yet.
% these would be subsections "Changes from v. WD-..."
% Use itemize environments.

\begin{admonition}{Note}
All dataset citations in this document were created using BibTeX. The BibTeX for the data/service/collection DOIs is based upon the \texttt{verbatim} metadata supplied by the archive to their choice of DOI registration services as expressed by the doi2bib service\footnote{\url{https://doi2bib.org}}. The doi2bib service was selected because it does the best job protecting unfielded author names from complete BibTeX annihilation.
Every other reference BibTeX came from NASA ADS\footnote{\url{https://ui.adsabs.harvard.edu/}}.
\end{admonition}

% NOTE: IVOA recommendations must be cited from docrepo rather than ivoabib
% (REC entries there are for legacy documents only)
\bibliography{ivoatex/ivoabib,ivoatex/docrepo,DOI4Archives}


\end{document}
